\section[Is there a clear correlation between the PSNR and SSIM; and does the SSIM better match the perceived quality.]{}
PSNR (Peak signal-to-noise ratio) is known for its inconsistency with human eye perception. The idea behind PSNR is that a noise signal is added to the original signal. It doesn't take into account things where humans are sensitive to (e.g. the structure in an image).
SSIM (Structural Similarity) however does take the human eye perception into account. To find the structural similarity of two images, it takes into account the luminance, contrast and structure (to which a human is sensitive).
\npar
In short: PSNR is based upon perceived errors in an image, while SSIM is based upon perceived change in structural information between 2 images. Thus, there is no clear correlation between the PSNR values and SSIM values. This can also be seen when comparing the graphs in Appendices \ref{app:Q9} and \ref{app:Q10}: when we look to the results with the simple pattern, exercise 3 (motion) seems to be the best regarding to the PSNR values (highest values). On the contrary, when we look to the SSIM values, exercise 2 (spatial) seems to yield the best results.